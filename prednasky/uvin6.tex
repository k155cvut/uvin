\documentclass[czech]{beamer}
\usepackage{helvet}
\renewcommand{\ttdefault}{lmtt}
\usepackage[T1]{fontenc}
\usepackage[utf8]{inputenc}

\makeatletter
%%%%%%%%%%%%%%%%%%%%%%%%%%%%%% Specific LaTeX commands.
% this default might be overridden by plain title style
\newcommand\makebeamertitle{\frame{\maketitle}}%
% (ERT) argument for the TOC
\AtBeginDocument{%
  \let\origtableofcontents=\tableofcontents
  \def\tableofcontents{\@ifnextchar[{\origtableofcontents}{\gobbletableofcontents}}
  \def\gobbletableofcontents#1{\origtableofcontents}
}
\newenvironment{lyxcode}
  {\par\begin{list}{}{
    \setlength{\rightmargin}{\leftmargin}
    \setlength{\listparindent}{0pt}% needed for AMS classes
    \raggedright
    \setlength{\itemsep}{0pt}
    \setlength{\parsep}{0pt}
    \normalfont\ttfamily}%
   \def\{{\char`\{}
   \def\}{\char`\}}
   \def\textasciitilde{\char`\~}
   \item[]}
  {\end{list}}

%%%%%%%%%%%%%%%%%%%%%%%%%%%%%% User specified LaTeX commands.
\date{}
\beamertemplateshadingbackground{red!5}{structure!5}

\usepackage{beamerthemecambridgeus}
\usepackage{pgfnodes,pgfarrows,pgfheaps}
\usepackage {multicol}

\beamertemplatetransparentcovereddynamicmedium

\makeatother

\usepackage{babel}
\begin{document}
\title{Funkce.}
\subtitle{Funkce, metody. Předávání parametrů. Mutable/Immutable objekty. Poziční
argumenty.}
\author{Tomáš Bayer | bayertom@fsv.cvut.cz}
\institute{Katedra geomatiky, fakulta stavební ČVUT.}
\makebeamertitle
\begin{frame}{Obsah přednášky}

\tableofcontents{}
\end{frame}


\section{Podprogram}
\begin{frame}{1. Procedura, funkce, metoda}

{\scriptsize Samostatná část programu konající specializovanou funkci. }{\scriptsize\par}

{\scriptsize Umístěny mimo hlavní program.}{\scriptsize\par}

{\scriptsize Lze opakovaně spouštět z hlavního programu či jiného podprogramu.\medskip{}
}{\scriptsize\par}

{\scriptsize Tvořeny: }{\scriptsize\par}
\begin{itemize}
\item {\scriptsize procedurami, }{\scriptsize\par}
\item {\scriptsize funkcemi, }{\scriptsize\par}
\item {\scriptsize metodami.\medskip{}
}{\scriptsize\par}
\end{itemize}
{\scriptsize\textbf{\textcolor{brown}{Výhoda~použití: }}}{\scriptsize\par}

{\scriptsize Zvýšení přehlednosti a čitelnosti programu.}{\scriptsize\par}

{\scriptsize Urychlují vývoj a ladění programu.}{\scriptsize\par}

{\scriptsize Možnost opakovaného provádění výpočtů.\bigskip{}
}{\scriptsize\par}

{\scriptsize\textbf{\textcolor{brown}{Volání (spuštění):}}}{\scriptsize\par}

{\scriptsize Volány se seznamem parametrů, kterým předány hodnoty potřebné
pro výpočet.}{\scriptsize\par}

{\scriptsize Většinou vrací nějaký výsledek, tzv. }{\scriptsize\emph{návratová
hodnota}}{\scriptsize . }{\scriptsize\par}

{\scriptsize Avšak nemusíme předávat žádné údaje popř. žádné výsledky
nevrací.\medskip{}
}{\scriptsize\par}

{\scriptsize\textbf{\textcolor{brown}{Procedura~vs.~funkce~vs.~metoda:}}}{\scriptsize\par}

{\scriptsize Některé jazyky nerozlišují funkce/procedury.}{\scriptsize\par}

{\scriptsize Procedura nevrací výsledek, funkce ano.}{\scriptsize\par}

{\scriptsize Metoda obdobou procedury/funkce v OOP.}{\scriptsize\par}
\end{frame}


\section{Funkce a její deklarace}
\begin{frame}{2. Funkce a její deklarace}

{\scriptsize Před použitím funkce nutná její deklarace.}{\scriptsize\par}

{\scriptsize Tvořena hlavičkou funkce a tělem funkce.}{\scriptsize\par}
\begin{lyxcode}
{\scriptsize def~jmeno\_funkce(form\_param1,~form\_param2,...):~}{\scriptsize\par}

{\scriptsize ~~~telo~funkce}{\scriptsize\par}

{\scriptsize\medskip{}
}{\scriptsize\par}
\end{lyxcode}
{\scriptsize\textbf{\textcolor{brown}{Hlavička funkce:}}}{\scriptsize\par}

{\scriptsize Jméno funkce, typ návratové hodnoty, seznam formálních
parametrů (FP).}{\scriptsize\par}

{\scriptsize V Pythonu netřeba uvádět datové typy (avšak můžeme, Type
Hints).}{\scriptsize\par}

{\scriptsize Python neumí přetěžovat funkce.}{\scriptsize\par}

{\scriptsize\medskip{}
}{\scriptsize\par}

{\scriptsize\textbf{\textcolor{brown}{Tělo funkce:}}}{\scriptsize\par}

{\scriptsize Představuje blok.}{\scriptsize\par}

{\scriptsize Obsahuje výkonný kód funkce.}{\scriptsize\par}

{\scriptsize\medskip{}
}{\scriptsize\par}

{\scriptsize Jméno (identifikátor) funkce zpravidla psáno malými písmeny.}{\scriptsize\par}

{\scriptsize Voleno tak, aby vyjadřovalo funkcionalitu.}{\scriptsize\par}

{\scriptsize\bigskip{}
}{\scriptsize\textbf{\textcolor{brown}{Podpora Type Hints: }}}{\scriptsize\par}

{\scriptsize Lze specifikovat typy formálních parametrů i návratový
typ, kontrola typů. }{\scriptsize\par}
\begin{lyxcode}
{\scriptsize def~jmeno\_funkce(form\_param1~:~typ,~form\_param2~:~typ,...)->typ:~\medskip{}
}{\scriptsize\par}
\end{lyxcode}
{\scriptsize Pouze informační charakter.}{\scriptsize\par}
\end{frame}

\begin{frame}{3. Volání funkce}

{\tiny = spuštění funkce, tvoří ho:}{\tiny\par}
\begin{itemize}
\item {\tiny jméno funkce, }{\tiny\par}
\item {\tiny seznam skutečných parametrů (SP).}{\tiny\par}
\end{itemize}
{\tiny\textbf{\textcolor{brown}{Funkce bez návratové hodnoty:}}}{\tiny\par}

{\tiny Volání funkce}{\tiny\par}
\begin{lyxcode}
{\tiny funkce(sp1,~sp2,...)~~~~~\#Pass~n~parameters}{\tiny\par}
\end{lyxcode}
{\tiny V některých jazycích deklarace funkce musí předcházet volání.}{\tiny\par}

{\tiny V Pythonu nehraje roli.\medskip{}
}{\tiny\par}

{\tiny\textbf{\textcolor{brown}{Funkce s návratovou hodnotou/hodnotami}}}{\tiny\emph{:}}{\tiny\par}

{\tiny Funkce v Pythonu schopna vrátit }{\tiny\emph{více hodnot, }}{\tiny netypické
pro většinu programovacích jazyků.}{\tiny\par}

{\tiny Volání součástí přiřazovacího příkazu}{\tiny\par}
\begin{lyxcode}
{\tiny prom1,~prom2,~prom3~=~funkce(sp1,~sp2,~sp3...)~~\#Return~3~parameters}{\tiny\par}

{\tiny prom~=~funkce(sp1,~sp2,~sp3,...)~~~~~~~~~~~~~~~~\#Return~n~parameters~as~tuple}{\tiny\par}
\end{lyxcode}
{\tiny V těle funkce klíčové slovo }{\tiny\texttt{return.}}{\tiny\par}

{\tiny Za ním uvedena/y hodnota/y vracená/é funkcí. }{\tiny\par}
\begin{lyxcode}
{\tiny def~addsubtract(a,~b):~~~~~~~~\#Declaration}~

{\tiny\par}{\tiny ~~~return~a+b,~a-b~~~~~~~~~~~~\#Return~addition/multiplication~of~2~variables~}{\tiny\par}

{\tiny ~~~a=7~~~~~~~~~~~~~~~~~~~~~~~~\#Unreachable~code}{\tiny\par}

{\tiny ..}{\tiny\par}

{\tiny x1~=~5~~~~~~~~~~~~~~~~~~~~~~~~\#First~variable}{\tiny\par}

{\tiny x2~=~7~~~~~~~~~~~~~~~~~~~~~~~~\#Second~variable}{\tiny\par}

{\tiny x3,~x4~=~addsubtract(x1,~x2)~~\#Function~call~and~assignment}{\tiny\par}

{\tiny >\textcompwordmark >\textcompwordmark >~12}{\tiny\par}
\end{lyxcode}
{\tiny Za klíčovým slovem }{\tiny\texttt{return}}{\tiny{} nesmí následovat
žádný kód. }{\tiny\par}

{\tiny Kód byl by nedostupný}{\tiny\emph{ $\Rightarrow$ }}{\tiny\textcolor{brown}{unreachable
kód}}{\tiny .}{\tiny\par}
\end{frame}

\begin{frame}{4. Funkce main()}

{\footnotesize Funkce volána při spuštění programu automaticky (C/C++,
Java).\medskip{}
}{\footnotesize\par}

{\footnotesize Můžeme jí předávat argumenty z příkazového řádku.}{\footnotesize\par}

{\footnotesize Nemá žádnou návratovou hodnotu.\medskip{}
}{\footnotesize\par}

{\footnotesize Výhodou lepší organizace a čitelnost kódu.}{\footnotesize\par}
\begin{lyxcode}
{\footnotesize def~main(args):~~\#Arguments~from~command~line}{\footnotesize\par}

{\footnotesize ~~~pass}{\footnotesize\par}
\end{lyxcode}
{\footnotesize Proměnné deklarované uvnitř }{\footnotesize\texttt{main()}}{\footnotesize{}
jsou lokální. }{\footnotesize\par}

{\footnotesize Proměnné deklarované vně }{\footnotesize\texttt{main()}}{\footnotesize{}
jsou globální (Python).}{\footnotesize\par}
\begin{lyxcode}
{\footnotesize if~\_\_name\_\_~==~'\_\_main\_\_':}{\footnotesize\par}

{\footnotesize ~~~main(sys.argv)}{\footnotesize\par}
\end{lyxcode}
{\footnotesize Soubor test.py lze používat dvěma způsoby:}{\footnotesize\par}
\begin{enumerate}
\item {\footnotesize Import ve formě modulu:}{\footnotesize\par}

{\footnotesize Pak }{\footnotesize\texttt{\_\_name\_\_= test}}{\footnotesize{}
a blok se nevykoná.}{\footnotesize\par}
\begin{lyxcode}
{\footnotesize import~test}{\footnotesize\par}
\end{lyxcode}
\item {\footnotesize Přímé spuštění:}{\footnotesize\par}

{\footnotesize Pak }{\footnotesize\texttt{\_\_name\_\_= \_\_name\_\_
}}{\footnotesize a blok se vykoná.}{\footnotesize\par}
\begin{lyxcode}
{\footnotesize python~test.py}{\footnotesize\par}
\end{lyxcode}
\end{enumerate}
\end{frame}

\begin{frame}{5. Formální a skutečné parametry funkce}

{\footnotesize\textbf{\textcolor{brown}{Skutečné~parametry:}}}{\footnotesize\par}

{\footnotesize Použity při volání funkce.}{\footnotesize\par}

{\footnotesize Jejich hodnoty předávány formálním parametrům.}{\footnotesize\par}

{\footnotesize\bigskip{}
}{\footnotesize\par}

{\footnotesize\textbf{\textcolor{brown}{Formální~parametry:}}}{\footnotesize\par}

{\footnotesize Lokální proměnné deklarovány v hlavičce funkce.}{\footnotesize\par}

{\footnotesize Vznikají v okamžiku volání funkce.}{\footnotesize\par}

{\footnotesize Zanikají při ukončení běhu funkce.\medskip{}
}{\footnotesize\par}

{\footnotesize FP stejného datového typu jako SP. }{\footnotesize\par}

{\footnotesize Jinak provedena implicitní konverze (nelze -li, chyba).}{\footnotesize\par}

{\footnotesize\bigskip{}
}{\footnotesize\par}

{\footnotesize Dva způsoby předávání formálních parametrů:}{\footnotesize\par}
\begin{itemize}
\item {\footnotesize\emph{hodnotou (Pass by Value)}}{\footnotesize\par}

{\footnotesize Předávána kopie hodnoty FP.}{\footnotesize\par}

{\footnotesize V Pythonu podporováno.}{\footnotesize\par}
\item {\footnotesize\emph{odkazem (Pass by Reference)}}{\footnotesize\par}

{\footnotesize Předávána reference (tj. ``originální'' proměnná).}{\footnotesize\par}

{\footnotesize Python nepodporuje.}{\footnotesize\par}

{\footnotesize Lze ``obejít''.}{\footnotesize\par}
\end{itemize}
\end{frame}


\section{Předávání parametrů }
\begin{frame}{6. Předávání parametrů hodnotou}

{\small FP kopií SP. }{\small\par}

{\small Jakákoliv změna hodnoty FP neovlivní hodnotu SP.}{\small\par}

{\small Originální data nelze ve funkci ``přepsat'' (read only).\bigskip{}
}{\small\par}

{\small V Pythonu FP předávány }{\small\textbf{\textcolor{brown}{pouze
hodnotou}}}{\small .}{\small\par}

{\small Předávání~hodnot SP do FP prováděno při volání metody.}{\small\par}

{\small Hodnoty jsou předávány }{\small\emph{postupně:}}{\small\medskip{}
}{\small\par}

{\small Hodnotě 1. FP předána hodnota 1. SP.}{\small\par}

{\small Hodnotě 2. FP předána hodnota 2. SP.}{\small\par}
\begin{lyxcode}
{\small form\_param1=skut\_param1;}{\small\par}

{\small form\_param2=skut\_param2;}{\small\par}
\end{lyxcode}
{\small ...}{\small\par}

{\small\medskip{}
V Pythonu trochu komplikovanější.}{\small\par}

{\small Odlišné chování pro immutable/mutable objects.\medskip{}
}{\small\par}

{\small Mutable objects: mohou být modifikovány v těle funkce.}{\small\par}

{\small Immutable objects: nemohou být modifikovány v těle funkce.}{\small\par}
\end{frame}


\section{Immutable/Mutable objects}
\begin{frame}[plain]{7. Immutable/Mutable objekty}

{\scriptsize V Pythonu vše objektem. }{\scriptsize\par}

{\scriptsize Každý objekt má unikátní identifikátor.}{\scriptsize\par}

{\scriptsize Zjištění identifikátoru objektu: příkaz }{\scriptsize\texttt{id()}}{\scriptsize\par}
\begin{lyxcode}
{\scriptsize x=1}{\scriptsize\par}

{\scriptsize id(x)~}{\scriptsize\par}

{\scriptsize >\textcompwordmark >\textcompwordmark >~1613424592~}{\scriptsize\par}
\end{lyxcode}
{\scriptsize\textbf{\textcolor{brown}{Mutable objekty:}}}{\scriptsize\par}

{\scriptsize Mohou měnit stav/obsah.}{\scriptsize\par}

{\scriptsize V Pythonu: }{\scriptsize\texttt{list, set, dict}}{\scriptsize .\medskip{}
}{\scriptsize\par}

{\scriptsize\textbf{\textcolor{brown}{Immutable objekty:}}}{\scriptsize\par}

{\scriptsize Nemohou měnit stav/obsah.}{\scriptsize\par}

{\scriptsize V Pythonu vše ostatní: }{\scriptsize\texttt{int, float,
complex, string, tuple, frozen set}}{\scriptsize .}{\scriptsize\par}

{\scriptsize Při pokus o ``modifikaci'' immutable vytvořen nový objekt.}{\scriptsize\par}
\begin{lyxcode}
{\scriptsize x=x+1}{\scriptsize\par}

{\scriptsize id(x)~}{\scriptsize\par}

{\scriptsize >\textcompwordmark >\textcompwordmark >~1613424608~~~~~~~~\#ID~has~changed,~x~modified~(immutable)}{\scriptsize\par}
\end{lyxcode}
{\scriptsize U mutable objektů nenastává:}{\scriptsize\par}
\begin{lyxcode}
{\scriptsize L={[}1,~2,~3{]}}{\scriptsize\par}

{\scriptsize id(L)~}{\scriptsize\par}

{\scriptsize >\textcompwordmark >\textcompwordmark >~22010856}{\scriptsize\par}

{\scriptsize L{[}0{]}=3~~}{\scriptsize\par}

{\scriptsize >\textcompwordmark >\textcompwordmark >id(L)~}{\scriptsize\par}

{\scriptsize >\textcompwordmark >\textcompwordmark >~22010856~~~~~~~~~~\#ID~unchanged,~L~modified~(mutable)}{\scriptsize\par}
\end{lyxcode}
{\scriptsize Odlišné chování těchto objektů jako parametrů funkcí.}{\scriptsize\par}
\end{frame}

\begin{frame}[plain]{8. Ukázka předávání SP do FP}

{\scriptsize Ukázka1: Funkce }{\scriptsize\texttt{dist()}}{\scriptsize ,
formální parametry x1, y1, x2, y2, IO,}{\scriptsize\par}
\begin{lyxcode}
{\scriptsize def~dist~(x1,~y1,~x2,~y2):~~~~~~~~~~~\#Evaluate~distance}~

{\scriptsize\par}{\scriptsize ~~~dx~=~x2~-~x1~~~~~~~~~~~~~~~~~~~~~~\#Coordinate~diff.~for~x}{\scriptsize\par}

{\scriptsize ~~~dy~=~y2~-~y1~~~~~~~~~~~~~~~~~~~~~~\#Coordinate~diff.~for~y}{\scriptsize\par}

{\scriptsize ~~~return~(dx{*}dx~+~dy{*}dy){*}{*}0.5~~~~~~~\#Return~distance}{\scriptsize\par}
\end{lyxcode}
{\scriptsize Použití TypeHint:}{\scriptsize\par}
\begin{lyxcode}
{\scriptsize def~dist~(x1~:~float,~y1~:~float,~x2~:~float,~y2~:~float)->float:~}{\scriptsize\par}
\end{lyxcode}
{\scriptsize Volání funkce:}{\scriptsize\par}
\begin{lyxcode}
{\scriptsize xa~=~0,~ya~=~0,~xb~=~10,~yb~=~10;}{\scriptsize\par}

{\scriptsize d~~=~dist(xa,~ya,~xb,~yb);}{\scriptsize\par}
\end{lyxcode}
{\scriptsize Přiřazení FP do SK:}{\scriptsize\par}
\begin{lyxcode}
{\scriptsize provede~se:~x1~=~xa,~y1~=~ya,~x2~=~xb,~y2~=~yb;~~}{\scriptsize\par}
\end{lyxcode}
{\scriptsize Ukázka 2: předání seznamu jako parametru, MU:}{\scriptsize\par}
\begin{lyxcode}
{\scriptsize def~f~(L):~~~~~~~~~~~~~~~~~~~~~~~~~~~\#Pass~list}~

{\scriptsize\par}{\scriptsize ~~~L.append(1)~~~~~~~~~~~~~~~~~~~~~~~\#Append~item~to~the~list}{\scriptsize\par}

{\scriptsize ~~~L.append(2)~~~~~~~~~~~~~~~~~~~~~~~\#Append~item~to~the~list}{\scriptsize\par}
\end{lyxcode}
{\scriptsize Volání funkce:}{\scriptsize\par}
\begin{lyxcode}
{\scriptsize S~=~{[}{]}~~~~~~~~~~~~~~~~~~~~~~~~~~~~~~~\#Create~empty~list}{\scriptsize\par}

{\scriptsize f(S)~~~~~~~~~~~~~~~~~~~~~~~~~~~~~~~~~\#Call~function~f}{\scriptsize\par}

{\scriptsize ...}{\scriptsize\par}
\end{lyxcode}
{\scriptsize Přiřazení FP do SP:}{\scriptsize\par}
\begin{lyxcode}
{\scriptsize provede~se:~L~=~S~}{\scriptsize\par}
\end{lyxcode}
\end{frame}

\begin{frame}{9. Předání IO/MO podrobněji...}

{\tiny IO/MO objects jako formální parametry předávány hodnotou.}{\tiny\par}

{\tiny\medskip{}
}{\tiny\par}

{\tiny\textbf{\textcolor{brown}{Immutable objekty:}}}{\tiny\par}

{\tiny Při pokusu o modifikaci IO v těle funkce se vytvoří jeho kopie.}{\tiny\par}

{\tiny Jeho změna neovlivní původní objekt.}{\tiny\par}
\begin{lyxcode}
{\tiny def~f(x):~~~~~~~~~~~~~~~~~~~~~~~def~f(x):~~}{\tiny\par}

{\tiny ~~~print(id(x))~~~~~~~~~~~~~~~~~~~~x=3~~~~~~~~~~~~\#Local~copy}{\tiny\par}

{\tiny ~~~~~~~~~~~~~~~~~~~~~~~~~~~~~~~~~~~print(id(x))~~~\#New~ID}{\tiny\par}

{\tiny ...~~~~~~~~~~~~~~~~~~~~~~~~~~~~~~}{\tiny\par}

{\tiny x~=~1~~~~~~~~~~~~~~~~~~~~~~~~~~~x~=~1~~~~~~~~~~~~}{\tiny\par}

{\tiny print(id(x))~~~~~~~~~~~~~~~~~~~~print(id(x))~}{\tiny\par}

{\tiny f(x)~~~~~~~~~~~~~~~~~~~~~~~~~~~~f(x)~~~~~~~~~~~~~~\#x~can~not~be~modified~inside~f}{\tiny\par}

{\tiny print(x)~~~~~~~~~~~~~~~~~~~~~~~~print(x)}{\tiny\par}

{\tiny >\textcompwordmark >~1946910832~~~~~~~~~~~~~~~~~~~>\textcompwordmark >~1946910824}{\tiny\par}

{\tiny >\textcompwordmark >~1946910832~~~~~~~~~~~~~~~~~~~>\textcompwordmark >~1946910832}{\tiny\par}

{\tiny >\textcompwordmark >~1~~~~~~~~~~~~~~~~~~~~~~~~~~~~>\textcompwordmark >~1~~~~~~~~~~~~~~\#Value~has~not~changed}{\tiny\par}
\end{lyxcode}
{\tiny\medskip{}
}{\tiny\par}

{\tiny\textbf{\textcolor{brown}{Mutable objekty:}}}{\tiny\par}

{\tiny Mohou měnit stav/obsah.}{\tiny\par}

{\tiny V těle funkce mohou být modifikovány.}{\tiny\par}
\begin{lyxcode}
{\tiny def~f(L):~~~~~~~~~~~~~~~~~~~~~~~def~f(L):~~}{\tiny\par}

{\tiny ~~~print(id(L))~~~~~~~~~~~~~~~~~~~~L{[}0{]}=3~~~~~~~~\#Modify~object}{\tiny\par}

{\tiny ~~~~~~~~~~~~~~~~~~~~~~~~~~~~~~~~~~~print(id(l))~~\#ID~has~not~changed}{\tiny\par}

{\tiny .}{\tiny\par}

{\tiny L~=~{[}1,~2{]}~~~~~~~~~~~~~~~~~~~~~~L~=~{[}1,~2{]}}{\tiny\par}

{\tiny print(id(L))~~~~~~~~~~~~~~~~~~~~print(id(L))~}{\tiny\par}

{\tiny f(L)~~~~~~~~~~~~~~~~~~~~~~~~~~~~f(L)~~~~~~~~~~~~~\#L~was~modified~inside~f}{\tiny\par}

{\tiny print(L)~~~~~~~~~~~~~~~~~~~~~~~~print(L)}{\tiny\par}

{\tiny >\textcompwordmark >~1946910817~~~~~~~~~~~~~~~~~~~>\textcompwordmark >~1946910817}{\tiny\par}

{\tiny >\textcompwordmark >~1946910817~~~~~~~~~~~~~~~~~~~>\textcompwordmark >~1946910817}{\tiny\par}

{\tiny >\textcompwordmark >~{[}1,~2{]}~~~~~~~~~~~~~~~~~~~~~~~>\textcompwordmark >~{[}3,~2{]}~~~~~~~~\#Values~have~changed}{\tiny\par}
\end{lyxcode}
\end{frame}

\begin{frame}{10. Předávání parametrů odkazem}

{\footnotesize Při předávání nevzniká kopie skutečného parametru.}{\footnotesize\par}

{\footnotesize Formální i skutečný parametr pak odkazují na stejnou
proměnnou}{\footnotesize\emph{.}}{\footnotesize\par}

{\footnotesize V Pythonu nepodporováno, lze však ``obejít'' s mutable
objekty.}{\footnotesize\par}

{\footnotesize\bigskip{}
}{\footnotesize\par}

{\footnotesize Změna hodnoty FP uvnitř fce změní hodnoty SP.}{\footnotesize\par}

{\footnotesize Možnost ``přepsání'' hodnot volajících dat.}{\footnotesize\par}

{\footnotesize Menší nároky na HW, netřeba vytvářet kopie objektů.}{\footnotesize\par}

{\footnotesize\bigskip{}
}{\footnotesize\par}

{\footnotesize Praktická realizace: ``obejít'' lze předání parametrů
v seznamu.}{\footnotesize\par}

{\footnotesize Ve skutečnosti opět předáváme referenci, modifikujeme
prvky seznamu.}{\footnotesize\par}

{\footnotesize\bigskip{}
}{\footnotesize\par}

{\footnotesize Ukázka: prohození dvou čísel (využití mutable)}{\footnotesize\par}
\begin{lyxcode}
{\footnotesize def~function~swap(x):}{\footnotesize\par}

{\footnotesize ~~~temp~=~x{[}0{]}~~~~~~~~\#Create~temp.~variable,~store~first}{\footnotesize\par}

{\footnotesize ~~~x{[}0{]}~=~x{[}1{]}~~~~~~~~\#Copy~second~to~first}{\footnotesize\par}

{\footnotesize ~~~x{[}1{]}~=~temp~~~~~~~~\#Copy~first}{\footnotesize\par}
\end{lyxcode}
{\footnotesize Volání funkce:}{\footnotesize\par}
\begin{lyxcode}
{\footnotesize a~=~{[}1,~2{]}~~~~~~~~~~~~\#Create~list~of~2~items}{\footnotesize\par}

{\footnotesize swap(a)~~~~~~~~~~~~~~~\#Swap~items}{\footnotesize\par}
\end{lyxcode}
\end{frame}


\section{Výchozí hodnoty parametrů}
\begin{frame}{11. Výchozí hodnoty parametrů}

{\scriptsize FP mohou mít definovány výchozí hodnoty.}{\scriptsize\par}

{\scriptsize Ukázka:}{\scriptsize\par}
\begin{lyxcode}
{\scriptsize def~dist1d~(dx,~dy~=~0):~~~~~~\#1~FP~is~default}{\scriptsize\par}

{\scriptsize ~~~return~(dx~{*}~dx~+~dy~{*}~dy){*}{*}0.5}{\scriptsize\par}
\end{lyxcode}
{\scriptsize Ukázka:}{\scriptsize\par}
\begin{lyxcode}
{\scriptsize def~dist2d~(dx~=~0,~dy~=~0):~~\#2~FPs~are~default}{\scriptsize\par}

{\scriptsize ~~~return~(dx~{*}~dx~+~dy~{*}~dy){*}{*}0.5}{\scriptsize\par}

\end{lyxcode}
{\scriptsize Funkce lze spustit s:}{\scriptsize\par}
\begin{enumerate}
\item {\scriptsize hodnotami skutečných parametrů}{\scriptsize\par}

{\scriptsize Mají vyšší prioritu než výchozí hodnoty FP.}{\scriptsize\par}
\begin{lyxcode}
{\scriptsize d~=~dist1d(5,~6)~\#dx~=~5,~dy~=~6}{\scriptsize\par}

{\scriptsize d~=~dist2d(5,~6)~\#dx~=~5,~dy~=~6}{\scriptsize\par}
\end{lyxcode}
\item {\scriptsize výchozími hodnotami formálních parametrů}{\scriptsize\par}

{\scriptsize SP dosazeny za FP:}{\scriptsize\par}
\begin{lyxcode}
{\scriptsize d~=~dist2d()~~~~~\#dx~=~0,~dy~=~0}{\scriptsize\par}
\end{lyxcode}
\item {\scriptsize Kombinace 1) + 2)}{\scriptsize\par}
\begin{lyxcode}
{\scriptsize d~=~dist1d(5)~~~~\#dx~=~5,~dy~=~0}{\scriptsize\par}

{\scriptsize d~=~dist2d(5)~~~~\#dx~=~5,~dy~=~0}{\scriptsize\par}
\end{lyxcode}
\end{enumerate}
\end{frame}


\section{Předávání funkcí jako parametrů}
\begin{frame}{12. Předávání funkcí jako parametrů}

{\scriptsize Funkce mohou být předávány jako parametry jiné funkci.}{\scriptsize\par}

{\scriptsize Předaná funkce volána se stejným počtem FP (podobný princip
jako u proměnných).}{\scriptsize\par}

{\scriptsize Tento mechanismus není běžný u jiných programovacích jazyků
(C++, pointer to function).\bigskip{}
}{\scriptsize\par}

{\scriptsize Příklad: Výpočet kombinačního čísla:
\[
C=\left(\begin{array}{c}
n\\
k
\end{array}\right)=\frac{n!}{(n-k)!k!}.
\]
}{\scriptsize\par}

{\scriptsize Faktoriál:}{\scriptsize\par}
\begin{lyxcode}
{\scriptsize def~fact(n):~~~~~~~~\#fact~has~one~parameter}{\scriptsize\par}

{\scriptsize ~~~f~=~1}{\scriptsize\par}

{\scriptsize ~~~while~n~>~1:}{\scriptsize\par}

{\scriptsize ~~~~~~f~{*}=~n~~~~~}{\scriptsize\par}

{\scriptsize ~~~~~~n~-=~1}{\scriptsize\par}

{\scriptsize ~~~return~f~~~}{\scriptsize\par}
\end{lyxcode}
{\scriptsize Kombinační číslo:}{\scriptsize\par}
\begin{lyxcode}
{\scriptsize def~comb(n,~k,~f)}{\scriptsize\par}

{\scriptsize ~~~return~f(n)~/~(f(n~-~k)~{*}~f(k))~~~\#f~has~1~parameter~as~fact}{\scriptsize\par}
\end{lyxcode}
{\scriptsize Volání:}{\scriptsize\par}
\begin{lyxcode}
{\scriptsize c=comb(5,~3,~fact)}{\scriptsize\par}
\end{lyxcode}
\end{frame}


\section{Poziční argumenty}
\begin{frame}[plain]{13. Poziční argumenty}

{\scriptsize V Pythonu může být libovolný počet FP zastoupen}{\scriptsize\par}
\begin{lyxcode}
{\scriptsize{*}args~~~~~{*}{*}kwargs}{\scriptsize\par}
\end{lyxcode}
{\scriptsize Použiváno v případě, kdy není znám počet FP.}{\scriptsize\par}

{\scriptsize Po předání vytvořena $n$-tice {*} nebo slovník {*}{*}.}{\scriptsize\par}
\begin{lyxcode}
{\scriptsize def~sum({*}nums):~~~~~~~~~~~~~~~~~~def~write({*}{*}items)}{\scriptsize\par}

{\scriptsize ~~~s~=~0~~~~~~~~~~~~~~~~~~~~~~~~~~~~~for~it~in~items:~}{\scriptsize\par}

{\scriptsize ~~~for~num~in~nums:~~~~~~~~~~~~~~~~~~~~~~print(it,~items{[}it{]})~~\#Key,~value}{\scriptsize\par}

{\scriptsize ~~~~~~s~+=~num~~~~~~~~}{\scriptsize\par}

{\scriptsize ~~~return~s~~~}{\scriptsize\par}
\end{lyxcode}
{\scriptsize Volání funkce:}{\scriptsize\par}
\begin{lyxcode}
{\scriptsize s~=~sum(1,~2,~3,~4)~~~~~~~~~~~~~~~~~~~~~~~~~~~\#Add~4~numbers}{\scriptsize\par}

{\scriptsize write~(name='Jan',~surname='Novak',~age=25)~~~\#Print~database}{\scriptsize\par}
\end{lyxcode}
{\scriptsize Poziční argumenty mohou být kombinovány s ``obyčejným''i:}{\scriptsize\par}
\begin{lyxcode}
{\scriptsize def~f(a,~b=100,~{*}c,~{*}{*}d):}{\scriptsize\par}

{\scriptsize ~~~...}{\scriptsize\par}
\end{lyxcode}
{\scriptsize Volání funkce:}{\scriptsize\par}
\begin{lyxcode}
{\scriptsize f(1,~2,~3,~4,~e=5,~f=6,~g=7)~~}{\scriptsize\par}
\end{lyxcode}
{\scriptsize Pak:}{\scriptsize\par}
\begin{lyxcode}
{\scriptsize >\textcompwordmark >a=1,~b=100,~c~=~(2,3,4),~d~=~\{e:5,~f:6,~g:7\}}{\scriptsize\par}
\end{lyxcode}
\end{frame}


\end{document}
