\documentclass[czech]{beamer}
\usepackage{helvet}
\renewcommand{\ttdefault}{lmtt}
\usepackage[T1]{fontenc}
\usepackage[utf8]{inputenc}
\usepackage{amstext}
\usepackage{amssymb}
\usepackage{stmaryrd}
\usepackage{undertilde}

\makeatletter
%%%%%%%%%%%%%%%%%%%%%%%%%%%%%% Specific LaTeX commands.
% this default might be overridden by plain title style
\newcommand\makebeamertitle{\frame{\maketitle}}%
% (ERT) argument for the TOC
\AtBeginDocument{%
  \let\origtableofcontents=\tableofcontents
  \def\tableofcontents{\@ifnextchar[{\origtableofcontents}{\gobbletableofcontents}}
  \def\gobbletableofcontents#1{\origtableofcontents}
}
\newenvironment{lyxcode}
  {\par\begin{list}{}{
    \setlength{\rightmargin}{\leftmargin}
    \setlength{\listparindent}{0pt}% needed for AMS classes
    \raggedright
    \setlength{\itemsep}{0pt}
    \setlength{\parsep}{0pt}
    \normalfont\ttfamily}%
   \def\{{\char`\{}
   \def\}{\char`\}}
   \def\textasciitilde{\char`\~}
   \item[]}
  {\end{list}}

%%%%%%%%%%%%%%%%%%%%%%%%%%%%%% User specified LaTeX commands.
\date{}
\beamertemplateshadingbackground{red!5}{structure!5}

\usepackage{beamerthemecambridgeus}
\usepackage{pgfnodes,pgfarrows,pgfheaps}
\usepackage {multicol}

\beamertemplatetransparentcovereddynamicmedium

\makeatother

\usepackage{babel}
\begin{document}
\title{Cykly.}
\subtitle{Cyklus for. Cyklus while. Break. Continue.}
\author{Tomáš Bayer | bayertom@fsv.cvut.cz}
\institute{Katedra geomatiky, fakulta stavební ČVUT.}
\makebeamertitle
\begin{frame}{Obsah přednášky}

\tableofcontents{}
\end{frame}

\section{Cykly}
\begin{frame}{1. Příkazy pro opakování}

Příkazy pro opakování nazývány iteračními příkazy.

Umožňují provádět příkaz/blok za splnění podmínky.\bigskip{}

V praxi nejčastěji realizovány prostřednictvím cyklů.

Vykytují se ve většině programovacích jazycích.\bigskip{}

Alternativou k cyklům rekurze.

\bigskip{}

\textbf{\textcolor{brown}{Řídící podmínka cyklu:}}

Představována booleovským výrazem.

Vyhodnocení: hodnota \texttt{true} nebo \texttt{false}.

\bigskip{}
Opakování prováděno v těle cyklu (blok, příkaz).\bigskip{}

Dle vstupní podmínky cykly děleny:
\begin{itemize}
\item s předem neznámým počtem opakování: \texttt{while},
\item s předem známým počtem opakování: \texttt{for}, \texttt{for-each},
\item s neznámým počtem opakování, alespoň 1x proběhne: \texttt{do while}.
\end{itemize}
\end{frame}


\section{Cyklus while}
\begin{frame}[plain]{2. Cyklus while}

{\scriptsize Používán, pokud počet opakování není předem znám.}{\scriptsize\par}

{\scriptsize Vyhodnocení řídící podmínky }{\scriptsize\emph{před}}{\scriptsize{}
průchodem cyklu. \bigskip{}
}{\scriptsize\par}

{\scriptsize Pokud nesplněna, tělo cyklu neproběhne. \bigskip{}
}{\scriptsize\par}

{\scriptsize Dokud splněna, opakovaně prováděno tělo cyklu.\medskip{}
}{\scriptsize\par}
\begin{lyxcode}
{\scriptsize while~vyraz:~~~~~~~~~\#~Ridici~podminka~cyklu}{\scriptsize\par}

{\scriptsize ~~~telo~cyklu~~~~~~~~\#~Pokud~rici~podminka~true}{\scriptsize\par}

{\scriptsize else:}{\scriptsize\par}

{\scriptsize ~~~telo~podminky~~~~~\#~Ridici~podminka~false}{\scriptsize\par}
\end{lyxcode}
{\scriptsize\medskip{}
V Pythonu specifická syntaxe, blok }{\scriptsize\texttt{else}}{\scriptsize :
pokud řídící podmínka }{\scriptsize\texttt{false}}{\scriptsize .}{\scriptsize\par}

{\scriptsize Ve většině jazyků podobná konstrukce chybí.\bigskip{}
}{\scriptsize\par}

{\scriptsize\textbf{\textcolor{brown}{Vnořený cyklus:}}}{\scriptsize\par}

{\scriptsize V těle cyklu další cyklus}{\scriptsize\par}
\begin{lyxcode}
{\scriptsize while~vyraz:~~~~~~~~~\#~Ridici~podminka~cyklu}{\scriptsize\par}

{\scriptsize ~~~while~vyraz:~}{\scriptsize\par}

{\scriptsize ~~~~~~telo~cyklu}{\scriptsize\par}
\end{lyxcode}
{\scriptsize\textbf{\textcolor{brown}{Nekonečný~cyklus:}}}{\scriptsize\par}

{\scriptsize V těle cyklu nutno modifikovat řídící proměnnou, jinak
nekonečný cyklus.}{\scriptsize\par}

{\scriptsize Pozor na tautologie a kontradikce!}{\scriptsize\par}
\end{frame}

\begin{frame}{3. Ukázka cyklu while: faktoriál}

Definice faktoriálu
\[
n!=\begin{cases}
n(n-1)!, & \mbox{pro }n>1,\\
1, & \mbox{pro }n=1.
\end{cases}
\]
Ukázka řešení v jazyce Python:
\begin{lyxcode}
n~=~10~~~~~~~~~~\#Initialize~n

f~=~1~~~~~~~~~~~\#Initialize~factorial

while~n~>~1:~~~~\#Repeat,~while~n~>~1~

~~~~~f~{*}=~n~~~~~\#Actualize~factorial

~~~~~n~-=~1~~~~~\#Decrement~n

print(f);~~~~~~~\#Print~result,~factorial.

~

>\textcompwordmark >\textcompwordmark >~3628800

~
\end{lyxcode}
\end{frame}


\section{Cyklus for}
\begin{frame}[plain]{4. Cyklus for}

{\scriptsize Používán, pokud je znám počet opakování.}{\scriptsize\par}

{\scriptsize\bigskip{}
Vyhodnocení řídící podmínky před každým provedením těla cyklu.}{\scriptsize\par}

{\scriptsize Tělo se prováděno dokud je pravdivá.}{\scriptsize\par}

{\scriptsize Pokud není pravdivý, skok do }{\scriptsize\texttt{else}}{\scriptsize{}
bloku.\bigskip{}
}{\scriptsize\par}

{\scriptsize Syntaxe v Pythonu odlišná od jiných programovacích jazyků
(}{\scriptsize\texttt{foreach}}{\scriptsize{} cyklus)}{\scriptsize\par}
\begin{lyxcode}
{\scriptsize for~prvek~in~iterovatelny~objekt:~~}{\scriptsize\par}

{\scriptsize ~~~~~~telo\_cyklu}{\scriptsize\par}

{\scriptsize else:}{\scriptsize\par}

{\scriptsize ~~~telo~podminky;}{\scriptsize\par}
\end{lyxcode}
{\scriptsize Iterace nad prvky datových struktury: {[}{]}, \{\}, ()
či intervalem.}{\scriptsize\par}

{\scriptsize\medskip{}
}{\scriptsize\par}

{\scriptsize\textbf{\textcolor{brown}{Iterovatelný objekt:}}}{\scriptsize\par}

{\scriptsize Objekt (dynamická datová struktura), můžeme přistupovat
k jeho prvkům.}{\scriptsize\par}

{\scriptsize Sekvenční procházení jeho prvků, přímý přístup.\bigskip{}
}{\scriptsize\par}

{\scriptsize Alternativně procházení prvků v nějakém intervalu, příkaz
}{\scriptsize\texttt{range}}{\scriptsize ().\bigskip{}
}{\scriptsize\par}

{\scriptsize\textbf{\textcolor{brown}{Vnořený cyklus:}}}{\scriptsize\par}

{\scriptsize V těle cyklu další cyklus}{\scriptsize\par}
\begin{lyxcode}
{\scriptsize for~prvek~in~in~iterovatelny~objekt:~~~~~~~~~\#~Ridici~podminka~cyklu}{\scriptsize\par}

{\scriptsize ~~~for~prvek~in~in~iterovatelny~objekt:~}{\scriptsize\par}

{\scriptsize ~~~~~~telo~cyklu}{\scriptsize\par}
\end{lyxcode}
\end{frame}

\begin{frame}[plain]{5. For v jiných programovacích jazycích}

{\footnotesize V jazycích C/C++/Java/C\# trochu jiná funkcionalita.}{\footnotesize\par}

{\footnotesize Chybí }{\footnotesize\texttt{else}}{\footnotesize{} blok.\bigskip{}
}{\footnotesize\par}

{\footnotesize Obecná syntaxe:}{\footnotesize\par}
\begin{lyxcode}
{\footnotesize for~(inicializacni\_vyraz;~testovaci\_vyraz;~zmenovy\_vyraz)}{\footnotesize\par}

{\footnotesize\{}{\footnotesize\par}

{\footnotesize ~~~~~~telo\_cyklu}{\footnotesize\par}

{\footnotesize\}}{\footnotesize\par}
\end{lyxcode}
{\footnotesize\textbf{\textcolor{brown}{Inicializační~výraz:}}}{\footnotesize\par}

{\footnotesize Je vykonán 1x před vyhodnocením testovacího výrazu\medskip{}
}{\footnotesize\par}

{\footnotesize\textbf{\textcolor{brown}{Testovací~výraz: }}}{\footnotesize\par}

{\footnotesize Určuje, zda se má provést tělo cyklu.}{\footnotesize\par}

{\footnotesize Prováděno, dokud hodnota testovacího výrazu }{\footnotesize\texttt{true}}{\footnotesize .
\medskip{}
}{\footnotesize\par}

{\footnotesize\textbf{\textcolor{brown}{Změnový~výraz:}}}{\footnotesize\par}

{\footnotesize Vyhodnocen na konci cyklu po vykonání těla cyklu.}{\footnotesize\par}

{\footnotesize Použit pro změnu hodnoty testovacího výrazu.\medskip{}
}{\footnotesize\par}

{\footnotesize Ukázka:}{\footnotesize\par}
\begin{lyxcode}
{\footnotesize for}~(int~i~=~0;~i~<~n;~i++)
\end{lyxcode}
\end{frame}

\begin{frame}[plain]{6. Ukázka cyklu for: hledání maxima, minima}

{\scriptsize Vstup $X=\{x_{i}\}_{i=1}^{n}.$}{\scriptsize\par}

{\scriptsize Hledáme
\[
\underline{x}=\min_{x_{i}\in X}(x_{i}),\qquad\overline{x}=\max_{x_{i}\in X}(x_{i}).
\]
Inicializace odhadů $\utilde{x}=x_{1}$, $\widetilde{x}=x_{1}$ prvním
prvkem, alternativně $\utilde{x}=\infty$, $\widetilde{x}=-\infty$.}{\scriptsize\par}

{\scriptsize Postupné porovnání $\utilde{x}$, $\widetilde{x}$ s $x_{i}$
\[
\utilde{x}=\min(\utilde{x},x_{i}),\qquad\widetilde{x}=\max(\widetilde{x},x_{i}).
\]
Pro průchodu celou posloupností 
\[
\underline{x}=\utilde{x},\qquad\overline{x}=\widetilde{x}.
\]
Ukázka řešení v jazyce Python:}{\scriptsize\par}
\begin{lyxcode}
{\scriptsize X~=~{[}10,~-27,~3,~48,~0,~95{]}~~~~~~~~~~~~~~\#Create~list}{\scriptsize\par}

{\scriptsize xmin~=~X{[}0{]}~~~~~~~~~~~~~~~~~~~~~~~~~~~~~~\#Initialize~minimum}{\scriptsize\par}

{\scriptsize xmax~=~X{[}0{]}~~~~~~~~~~~~~~~~~~~~~~~~~~~~~~\#Initialize~maximum}{\scriptsize\par}

{\scriptsize for~x~in~X:~~~~~~~~~~~~~~~~~~~~~~~~~~~~~~\#Repeat~for~$\forall x\in X$}{\scriptsize\par}

{\scriptsize ~~~~if~x~<~xmin:~~~~~~~~~~~~~~~~~~~~~~~~~\#Compare~x{[}i{]}~to~xmin}{\scriptsize\par}

{\scriptsize ~~~~~~~~xmin~=~x~~~~~~~~~~~~~~~~~~~~~~~~~\#Actualize~xmin}{\scriptsize\par}

{\scriptsize ~~~~if~x~>~xmax:~~~~~~~~~~~~~~~~~~~~~~~~~\#Compare~x{[}i{]}~to~xmax}{\scriptsize\par}

{\scriptsize ~~~~~~~~xmax~=~x~~~~~~~~~~~~~~~~~~~~~~~~~\#Actualize~xmax~~~~~~~~~~~~~~~~~~~~~~~~~~~~~~~~~~}{\scriptsize\par}

{\scriptsize print(xmax)~~~~~~~~~~~~~~~~~~~~~~~~~~~~~~\#Print~results}{\scriptsize\par}

{\scriptsize print(xmim)~~}{\scriptsize\par}

{\scriptsize >\textcompwordmark >\textcompwordmark >~95,~}{\scriptsize\par}

{\scriptsize >\textcompwordmark >\textcompwordmark >~-27}{\scriptsize\par}
\end{lyxcode}
\end{frame}

\begin{frame}[plain]{7. Ukázka cyklu for: bankomat }

{\tiny Vstup: částka $c$, $c\in\mathbb{N}$.}{\tiny\par}

{\tiny Výstup: vyplacení co nejmenším počtem bankovek/mincí
\[
B=\{1,2,5,...,2000,5000\}.
\]
Použití celočíselného dělení $\sslash$ a operace $\pmod{}$.}{\tiny\par}

{\tiny Počet bankovek
\[
n=c\sslash b.
\]
Zbytek částky k vyplacení
\[
c\equiv c\pmod b=c-nb.
\]
Postupně opakujeme poslední dva kroky.\medskip{}
}{\tiny\par}

{\tiny Ukázka řešení v jazyce Python:}{\tiny\par}
\begin{lyxcode}
{\tiny B={[}5000,2000,1000,500,200,100,50,20,10,5,2,1{]}~~}{\tiny\par}

{\tiny c~=~int(input(\textquotedbl Zadej~castku:~\textquotedbl ))}{\tiny\par}

{\tiny for~b~in~B:~~~~~~~~~~~~~~~~~~~~~~~~~~~~\#Opakuj~pro~kazdou~bankovku/minci}{\tiny\par}

{\tiny ~~~~n~=~c~//~b~~~~~~~~~~~~~~~~~~~~~~~~~\#Urci~pocet~bankovek}{\tiny\par}

{\tiny ~~~~c~=~c\%b~~~~~~~~~~~~~~~~~~~~~~~~~~~~\#Zbytkova~castka:~c~=~c~-~n~{*}~b~}{\tiny\par}

{\tiny ~~~~if~c~>~0:~~~~~~~~~~~~~~~~~~~~~~~~~~\#Pokud~jeste~neco~zbyva}{\tiny\par}

{\tiny ~~~~~~~~print(str(n)~+~\textquotedbl ~x~\textquotedbl ~+~str(b))~\#Vytiskni~pocet~bankovek}{\tiny\par}

{\tiny .}{\tiny\par}

{\tiny >\textcompwordmark >\textcompwordmark >~Zadej~castku:~2345}{\tiny\par}

{\tiny >\textcompwordmark >\textcompwordmark >~1~x~2000}{\tiny\par}

{\tiny >\textcompwordmark >\textcompwordmark >~1~x~200}{\tiny\par}

{\tiny >\textcompwordmark >\textcompwordmark >~1~x~100}{\tiny\par}

{\tiny >\textcompwordmark >\textcompwordmark >~2~x~20}{\tiny\par}

{\tiny >\textcompwordmark >\textcompwordmark >~1~x~5}{\tiny\par}
\end{lyxcode}
\end{frame}

\begin{frame}[plain]{8. List/Set/Dir comprehension}

{\footnotesize Využití }{\footnotesize\texttt{for}}{\footnotesize{} při
konstrukci nových dynamických struktur z existujících.}{\footnotesize\par}

{\footnotesize Speciální (úspornější) notace, umožňuje efektivní konstrukci.}{\footnotesize\par}

{\footnotesize Změnový výraz představuje libovolný Pythonovský výraz.\bigskip{}
}{\footnotesize\par}

{\footnotesize\textbf{List comprehension:}}{\footnotesize\par}

{\footnotesize Vytvoření nového seznamu z existujícího (nemění se).}{\footnotesize\par}
\begin{lyxcode}
{\footnotesize L~={[}1,~2,~3,~4,~5{]}}{\footnotesize\par}

{\footnotesize L2~=~{[}10~{*}~l~for~l~in~L{]}}{\footnotesize\par}

{\footnotesize >\textcompwordmark >\textcompwordmark >~{[}10,~20,~30,~40,~50{]}}{\footnotesize\par}
\end{lyxcode}
{\footnotesize\textbf{Set comprehension:}}{\footnotesize\par}

{\footnotesize Vytvoření nového setu z existujícího (nemění se).}{\footnotesize\par}
\begin{lyxcode}
{\footnotesize S~=\{1,~2,~3,~4,~5\}}{\footnotesize\par}

{\footnotesize S2~=~\{10~{*}~s~for~s~in~S\}}{\footnotesize\par}

{\footnotesize >\textcompwordmark >\textcompwordmark >~\{10,~20,~30,~40,~50\}}{\footnotesize\par}
\end{lyxcode}
{\footnotesize\textbf{Directory comprehension:}}{\footnotesize\par}

{\footnotesize Vytvoření nového seznamu z původního (nemění se).}{\footnotesize\par}

{\footnotesize Lze modifikovat key i value.}{\footnotesize\par}
\begin{lyxcode}
{\footnotesize D~=\{'a':1,~'b':2,~'c':3,~'d':4,~'e':5\}}{\footnotesize\par}

{\footnotesize D2~=~\{key:10~{*}~value~for~key,~value~in~D.items()\}}{\footnotesize\par}

{\footnotesize >\textcompwordmark >\textcompwordmark >~\{'a':~10,~'b':~20,~'c':~30,~'d':~40,~'e':~50\}}{\footnotesize\par}
\end{lyxcode}
\end{frame}


\section{Příkaz break()}
\begin{frame}[plain]{9. Příkaz break()}

{\small Okamžité (předčasné) ukončení provádění těla cyklu.}{\small\par}

{\small Provedeno, pokud podmínka vyhodnocena jako pravdivá.}{\small\par}
\begin{lyxcode}
{\small if~vyraz:}{\small\par}

{\small ~~~break;}{\small\par}
\end{lyxcode}
{\small Pokračování prvním příkazem následujícím za cyklem.}{\small\par}

{\small U vnořeného cyklu skok o 1 úroveň výše.\bigskip{}
}{\small\par}

{\small Používán, pokud byla splněna nějaká, a není třeba pokračovat.}{\small\par}

{\small Tuto podmínku nelze otestovat před průchodem těla cyklu}.

{\small\bigskip{}
}{\small\par}

{\small Příklad: nalezení prvku }{\small\texttt{val}}{\small{} v množině
}{\small\texttt{X:}}{\small\par}
\begin{lyxcode}
{\small found~=~False~~~~~~~~\#Test,~whether~x~in~X}{\small\par}

{\small for~x~in~X:~~~~~~~~~~\#Test~x~one~by~one}{\small\par}

{\small ~~~if~(x~==~val)~~~~~\#Element~val~found~in~X}{\small\par}

{\small ~~~~~~~found~=~true}{\small\par}

{\small ~~~~~~~break}{\small\par}
\end{lyxcode}
{\small\textbf{\textcolor{brown}{Výhoda použití:}}}{\small\par}

{\small Zjednodušení konstrukce řídící podmínky cyklu.}{\small\par}

{\small Část podmínky nemusí být v řídícím výrazu cyklu.}{\small\par}
\end{frame}


\section{Příkaz continue()}
\begin{frame}[plain]{10. Příkaz continue()}

{\footnotesize Přeskočení zbývající části příkazů v těle cyklu.}{\footnotesize\par}

{\footnotesize Provede přechod na další iteraci, pokud podmínka pravdivá.}{\footnotesize\par}
\begin{lyxcode}
{\footnotesize if~vyraz:}{\footnotesize\par}

{\footnotesize ~~~continue;}{\footnotesize\par}
\end{lyxcode}
{\footnotesize Příklad, třídící algoritmus Merge Sort:}{\footnotesize\par}
\begin{lyxcode}
{\footnotesize while~i~<~n~+~m):}{\footnotesize\par}

{\footnotesize ~~~~~~if~i~==~n~:~~~~~~\#Seznam~a~uz~nema~dalsi~prvky}{\footnotesize\par}

{\footnotesize ~~~~~~~~~c{[}k{]}~=~b{[}j{]}}{\footnotesize\par}

{\footnotesize ~~~~~~~~~j~=~j+1}{\footnotesize\par}

{\footnotesize ~~~~~~~~~continue~~~~~~\#Skok~na~dalsi~iteraci}{\footnotesize\par}

{\footnotesize ~~~~~~}{\footnotesize\par}

{\footnotesize ~~~~~~if~(j~==~m):~~~~~\#Seznam~b~uz~nema~dalsi~prvky}{\footnotesize\par}

{\footnotesize ~~~~~~~~~c{[}k{]}~=~a{[}i{]};}{\footnotesize\par}

{\footnotesize ~~~~~~~~~i~=~i~+~1}{\footnotesize\par}

{\footnotesize ~~~~~~~~~continue~~~~~~\#Skok~na~dalsi~iteraci}{\footnotesize\par}
\end{lyxcode}
{\footnotesize V praxi se příliš často nepoužívá, znepřehledňuje tělo
cyklu.\bigskip{}
}{\footnotesize\par}

{\footnotesize Nekombinovat více příkazů }{\footnotesize\texttt{continue}}{\footnotesize{}
v jednom těle (max 2), nepřehledné.\bigskip{}
}{\footnotesize\par}

{\footnotesize Skok v algoritmu, chyba v návrhu!}{\footnotesize\par}
\end{frame}


\section{Příkaz range()}
\begin{frame}[plain]{11. Příkaz range()}

{\footnotesize Vytváří celočíselnou aritmetickou posloupnost $(a_{n})$,
když $\exists d\in\mathbb{N}$, $\forall d\in\mathbb{N}$ a platí
\[
a_{n}=a_{1}+(n-1)d,
\]
$d$ je diference posloupnosti.}{\footnotesize\par}
\begin{lyxcode}
{\footnotesize ~~~range($a_{1}$,~}{\footnotesize\textrm{$a_{n}$}}{\footnotesize ,~$d$);~~~~~$a_{1}+kd<a_{n},\qquad d\in\mathbb{N}.$}{\footnotesize\par}
\end{lyxcode}
{\footnotesize Ukázka:}{\footnotesize\par}
\begin{lyxcode}
{\footnotesize X=range(1,~20,~2)}{\footnotesize\par}

{\footnotesize for~x~in~X:~~~}{\footnotesize\par}

{\footnotesize ~~~~print(x)}{\footnotesize\par}

{\footnotesize >\textcompwordmark >\textcompwordmark >~1~3~5~7~9~11~13~15~17~19}{\footnotesize\par}
\end{lyxcode}
{\footnotesize Často kombinován s cyklem }{\footnotesize\texttt{for}}{\footnotesize ,
tvorba celočíselného indexu}{\footnotesize\par}
\begin{lyxcode}
{\footnotesize range($i_{1}$,~}{\footnotesize\textrm{$i_{n}$}}{\footnotesize ,~$1$);~}{\footnotesize\par}
\end{lyxcode}
{\footnotesize Generuje posloupnost indexů $[i]$
\[
(i_{1},i_{1}+1,...,i_{1}+n-1).
\]
}{\footnotesize\par}

{\footnotesize Přes $a_{i}$ lze iterovat, 1. a 3. parametr nepovinný.\medskip{}
}{\footnotesize\par}

{\footnotesize Příklady použití:}{\footnotesize\par}
\begin{lyxcode}
{\footnotesize for~i~in~range(max):~~~~~~\#Interval~0,~max-1}{\footnotesize\par}

{\footnotesize ~~~pass;}{\footnotesize\par}

{\footnotesize for~i~in~range(min,~max):~\#Interval~min,~max-1}{\footnotesize\par}

{\footnotesize ~~~pass;}{\footnotesize\par}
\end{lyxcode}
\end{frame}

\begin{frame}{12. Ukázka použití příkazu continue}

Generování 10 náhodných čísel $x\in\left\langle x_{min},x_{max}\right\rangle $
v $\mathbb{R}$, $x_{min}=-x_{max}.$

Výpočet odmocniny
\[
y=\sqrt{x},\qquad x\geq0.
\]
Ukázka řešení v jazyce Python:
\begin{lyxcode}
import~random~~~~~~~~~~~~~~~~~~~~~~~\#Import~libraries

import~math

max~=~1000~~~~~~~~~~~~~~~~~~~~~~~~~~\#Maximum/minimum~value

for~i~in~range(10):~~~~~~~~~~~~~~~~~\#i=0,1,...,~9

~~~~x~=~random.randint(-max,~max)~~~\#Create~random~number~x

~~~~if~x~<~0:~~~~~~~~~~~~~~~~~~~~~~~\#x~is~negative

~~~~~~~~continue~~~~~~~~~~~~~~~~~~~~\#Skip~next~steps

~~~~y~=~math.sqrt(x)~~~~~~~~~~~~~~~~\#Compute~y=sqrt(x)

~~~~print(y)~~~~~~~~~~~~~~~~~~~~~~~~\#Print~value~y
\end{lyxcode}
\end{frame}

\begin{frame}{13. Zásady pro práci s cykly}

1) Cyklus by měl mít pouze jednu řídící proměnnou.\bigskip{}

2) Hodnota řídící proměnné u for by neměla být ovlivňována v těle
cyklu.\bigskip{}

3) Řídící podmínku cyklu zapisovat v kladném tvaru.

\bigskip{}
4) Řídící podmínku zjednodušovat.

\bigskip{}

5) Vyvarovat se vzniku nekonečných cyklů.\bigskip{}

6) Preference cyklů \texttt{while}, \texttt{for} před \texttt{do}
\texttt{while} (přehlednost).\bigskip{}

7) Příkaz \texttt{break} v těle cyklu používat pouze na jednom místě.\bigskip{}

8) ... Nebo lépe se mu zcela vyhnout :-).\bigskip{}

9) V těle cyklu nepoužívat skoky nebo jen výjimečně (\texttt{continue}).
\end{frame}


\end{document}
